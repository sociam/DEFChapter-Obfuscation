% PLEASE USE THIS FILE AS A TEMPLATE FOR THE PUBLICATION 
% Check file IOS-Book-Article.tex
%

\documentclass{IOS-Book-Article}     %[seceqn,secfloat,secthm]
%\usepackage{mathptmx}
%\usepackage[T1]{fontenc}
%\usepackage{times}%
%
%%%%%%%%%%% Put your definitions here


%%%%%%%%%%% End of definitions
\begin{document}
\begin{frontmatter}          % The preamble begins here.
%
%\pretitle{}
\title{Social Palimpsests - clouding the lens of the personal panopticon}
\runningtitle{\ldots or ``Suck my tailpipe'' - how I learned to stop worrying
and toxify my exhaust data}
%\subtitle{}

% Two or more authors:
\author[A]{\fnms{Dave} \snm{Murray-Rust}},
\author[B]{\fnms{Max} \snm{van Kleek}}
\author[B]{\fnms{Laura} \snm{Dragan}}
\runningauthor{Murray-Rust et. al.}
\address[A]{Centre for Intelligent Systems and Applications,
Department of Informatics, University of Edinburgh}
\address[B]{Southampton}

\begin{abstract}
The use of personal data has incredible potential to benefit both society and
individuals, through increased understanding of behaviour, communication and
support for emerging forms of socialisation and connectedness. However, there
are risks associated with disclosing personal information, and present systems
show a systematic asymmetry between the subjects of the data and those who
control and manage the way that data is propagated and used. In this chapter, we
explore a set of techniques for ameliorating the tension between the desire for
the benefits of sharing and a distrust of those with whom we share our data.
\end{abstract}

\begin{keyword}

\end{keyword}

\end{frontmatter}

%%%%%%%%%%% The article body starts:

\section*{Introduction}

\subsection{The rise of personal data and services reliant on it}

\subsection{Data, Capta, Fiat Data, Exhaust Data - and the impossibility
of limiting data diffusion}

``Refusal is not a practical option, as data collection is an inherent condition
of many essential societal transactions''
\cite{brunton2011vernacular}

\subsection{The case for lying and the importance of anonymity}

\subsection{There's lots of great stuff out there!} 

Types of disinformation \cite{alexander2010Disinformation}:
\begin{description}
  \item[redaction] is hiding some or all of the information in a message
  \item[airbrushing] is changing some of the information. \emph{local crowd
  blending} means change it to a nearby message likely to be plausible.
  \emph{global crowd blending} means change it to a message in a dense part of
  the space.
  \item[curveball] add extra distracting information, push message into low
  density space
\end{description}

\subsection{Why doing it socially is difficult}

\subsection{Scenario}

Lets take the scenario of modifying location data; might want to:
\begin{itemize}
  \item Hide the fact we're away from home so we don't get robbed - create
  fictitious data that looks like it came from home town. Enlist friends.
  \item Avoid paparazzi - generate chaff so real location is obscured
  \item Disguise the fact we're going to ballet lessons - systematically lie,
  saying we're going to the boxing gym instead
\end{itemize}

Alexander's taxonomy \cite{alexander2010Disinformation} discusses several types
of disinformation which relate to modifying single messages. In contrast, due to
the pervasiveness of modern communications, we are concerned with modifying
message \emph{streams}, where a trace of multiple values must be considered.
Additionally, there is the possibility of interaction with others, whether it is
collusion to strengthen obfuscatory practices, or the addition---purposeful or
otherwise---of information which exposes the obfuscation.

It is problematic to consider the obfuscatory tactics here without a sense of
the scenario in which they are to be deployed. Our scenario in this paper is:
\begin{quote}
The user wishes to make use of services which expect location information; 
the location information provided is shared publicly and is almost
certainly stored indefinitely. At times, the user may want to draw on location
based information---such as restaurant recommendations or directions---and there
may be times when they with to verify that they were at a particular location.
\end{quote}

In order to illustrate the effects of these different strategies, we plot a one
dimensional representation of ``location'' against time.
\section{Tools}

\subsection{Personal Data Stores - allies on the intimacy battleground}

\subsection{Verification and provenance}




%%%%%%%%%%% The bibliography starts:
%\begin{thebibliography}{99}
%\bibitem{r1}
%\end{thebibliography}

\bibliographystyle{abbrv}
\bibliography{palimpsests}


\end{document}
