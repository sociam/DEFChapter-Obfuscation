% PLEASE USE THIS FILE AS A TEMPLATE FOR THE PUBLICATION 
% Check file IOS-Book-Article.tex
%

\documentclass{IOS-Book-Article}     %[seceqn,secfloat,secthm]
%\usepackage{mathptmx}
%\usepackage[T1]{fontenc}
%\usepackage{times}%
%
%%%%%%%%%%% Put your definitions here
\usepackage{paralist}
\usepackage{graphicx}
\usepackage{color}


%Make figures easier
\newcommand{\fig}[3][0.9]{
\begin{figure}[tp]
\begin{center}
\includegraphics[width=#1\textwidth]{figures/#2}
\caption{#3}
\label{fig:#2}
\end{center}
\end{figure}
}

\newcommand{\tbox}[3][red]{{
\color{#1}\noindent{
   \fbox{\scriptsize{ {\bf #2} \textsl{#3}}}
   \vspace{2pt}
}
}}


\definecolor{darkgreen}{rgb}{0,0.4,0}
\newcommand{\todo}[1]{\tbox{TODO:}{#1}}

%%%%%%%%%%% End of definitions
\begin{document}
\begin{frontmatter}          % The preamble begins here.
%
%\pretitle{}
\title{Social Palimpsests - clouding the lens of the personal panopticon}
\runningtitle{\ldots or ``Suck my tailpipe'' - how I learned to stop worrying
and toxify my exhaust data}
%\subtitle{}

% Two or more authors:
\author[A]{\fnms{Dave} \snm{Murray-Rust}},
\author[B]{\fnms{Max} \snm{van Kleek}}
\author[B]{\fnms{Laura} \snm{Dragan}}
\runningauthor{Murray-Rust et. al.}
\address[A]{Centre for Intelligent Systems and Applications,
Department of Informatics, University of Edinburgh}
\address[B]{Southampton}

\begin{abstract}
The use of personal data has incredible potential to benefit both society and
individuals, through increased understanding of behaviour, communication and
support for emerging forms of socialisation and connectedness. However, there
are risks associated with disclosing personal information, and present systems
show a systematic asymmetry between the subjects of the data and those who
control and manage the way that data is propagated and used. In this chapter, we
explore a set of techniques for ameliorating the tension between the desire for
the benefits of sharing and a distrust of those with whom we share our data.
\end{abstract}

\begin{keyword}
Obfuscation; Data politics; Personal Data Stores; Social Machines; Pants-on-fire;
\end{keyword}

\end{frontmatter}

%%%%%%%%%%% The article body starts:

\section*{Introduction}

\subsection{The rise of personal data and services reliant on it}


``Refusal is not a practical option, as data collection is an inherent condition
of many essential societal transactions''
\cite{brunton2011vernacular}

\subsection{Data, Capta, Fiat Data, Exhaust Data - and the impossibility
of limiting data diffusion}

Increasingly, in order to utilise services, we must provide our data to third
parties. This ranges from mobile phone numbers being required for Yahoo
accounts, to location data being shared with Foursquare or Grindr, to the NHS
adding personal health information to centralised databases. We call this fiat
data - an organisation uses its position to demand (by fiat) the disclosure of
certain information in return for use of its services. In some cases, this is a
necessary requirement for the service to be worthwhile, but in many cases it
represents an attempt by the organisation to create a monetizable product from its users.

Sharing data, by definition, is the entrusting of other parties with
information; this necessarily involves relinquishing control over how it is
subsequently handled and disseminated. However, data is persistent, while people
and contexts change. A government may decide to share previously confidential
data, as in the case of the recent care.data fiasco in the UK; a company can be
bought and its assets acquired–the purchase of Moves by Facebook raised issues
around the terms and conditions of data handling companies; and even without
malice accidents can expose vast swathes of personal data, or court proceedings
may force private communications to become public - the Enron emails still
represent the largest publicly available corpus of private emails.  

\subsection{Multiple Identity}

A natural part of online life is the ability to tailor the persona we present to
different communities and contexts. An individual may want to disclose certain
things to their professional colleagues, while presenting differently to friends
and family or non-mainstream friend groups. Sharing certain personal data is a
barrier to this, as its basis in physical fact provides multiple opportunities
for joining up otherwise separate databases. Most, if not all, social
interactions involve both strategic omissions and various kinds of lies and
non-truths to manage the myriad conflicting social demands placed upon us. The
lie maintenance required to avoid discovery may be trivial (``sorry, I’m hungry,
have to go!'') but may become significantly more complicated as lies extend over
time, and become woven into the social fabric.

\subsection{Why doing it socially is difficult}

\subsection{The case for lying and the importance of anonymity}

\subsection{Personal Data Stores - allies on the intimacy battleground}

Personal data stores (PDS) represent a partial solution to issue of
presentation: having trusted, user controlled repositories for data enables a
more user-centric approach to management of capta---those data which we choose
to take and preserve. Bridges can then be built between personal data stores and
the rest of the world in order to support the connected, networked interactions
which users now expect. If these bridges simply share the data, even in a
controlled manner, nothing has been gained; hence the bridges become conduits
for manipulating truth and constructing falsehoods. As personal data stores
accumulate more real-time contextual data about the individual, as well as about
the individual’s social connections, PDSes can provide support for the often
stressful and mentally burdensome task of lie maintenance, for example: i)
identifying when a person's real activities or whereabouts contradict a lie, and
might be discovered; ii) identifying indirect social channels that could expose
a lie (e.g. through friends of friends); iii) suggesting appropriate lies to use
which are least likely to be detected; iv) suggesting individuals to lie to to
support lie maintenance (e.g. friends of the person being lied to)

\subsection{Why verification and provenance are better than sharing}


Sharing is a crude mechanism. Once data has been shared, the originator can no
longer exert control over it, and must rely on the behaviour of the recipient,
which as noted may fail to meet user expectations. Validation, however is a more
subtle tool: if a user’s personal dataset can be made sufficiently questionable
as to be useless on its own, then locus of control shifts to the user choosing
to validate parts of the dataset, which can be performed in a more nuanced,
contextualised manner. If a user is the final arbiter of trust, they can decide
to i) sign parts of their record, so that it is verified public fact; ii)
co-sign it with another entity, so either can verify it but not anyone else;
iii) verify it through an anonymous channel, so that the entity to whom they
provide verification cannot propagate the claim further. This verification can
be carried out entirely separately from the datastore itself, allowing for the
presentation of different datasets as valid  in different contexts, as well as
unorthodox methods such as using the Bitcoin blockchain to notarise datasets, so
that they can be verified in the future without revealing them as true at the
time.

\section{Review of current approaches and tools} 

Types of disinformation \cite{alexander2010Disinformation}:
\begin{description}
  \item[redaction] is hiding some or all of the information in a message
  \item[airbrushing] is changing some of the information. \emph{local crowd
  blending} means change it to a nearby message likely to be plausible.
  \emph{global crowd blending} means change it to a message in a dense part of
  the space.
  \item[curveball] add extra distracting information, push message into low
  density space
\end{description}






\section{A selection of PDS enabled obfuscation strategies}

Alexander's taxonomy \cite{alexander2010Disinformation} discusses several types
of disinformation which relate to modifying single messages. In contrast, due to
the pervasiveness of modern communications, we are concerned with modifying
message \emph{streams}, where a trace of multiple values must be considered.
Additionally, there is the possibility of interaction with others, whether it is
collusion to strengthen obfuscatory practices, or the addition---purposeful or
otherwise---of information which exposes the obfuscation.

It is problematic to consider the obfuscatory tactics here without a sense of
the scenario in which they are to be deployed. Our scenario in this paper is:
\begin{quote}
The user wishes to make use of services which expect location information; 
the location information provided is shared publicly and is almost
certainly stored indefinitely. At times, the user may want to draw on location
based information---such as restaurant recommendations or directions---and there
may be times when they with to verify that they were at a particular location.
\end{quote}
The service is hence \emph{semi-trusted}: there are some benefits which the user
wishes to accrue, but there are aspects of the service which makes the user
unwilling to entrust their complete life history to it.

\fig{Mediation}{Models of interaction with semitrusted services. a) Direct
transmission of information; b) computationally mediated transmission, where a
personal data store is enlisted to aid in obfuscatory processes.}

The standard model of interaction (Figure \ref{fig:Mediation}a) involves the
user submitting their data directly to the service; for our obfuscatory
techniques, we would like to enlist computational support (Figure
\ref{fig:Mediation}b). This typified, but not limited to mediation from a PDS
which acts on behalf of the user to modify the data which they provide.

\subsection{Single Player Strategies}

\fig[1.05]{SinglePlayerObfuscation}{Obfuscation strategies for the lone agent}
Figure \ref{fig:SinglePlayerObfuscation} lists a collection of possible
obfuscation strategies. In all cases, a fictitious one-dimensional ``location"
measurement is plotted against time, to give a sense of how an
individual's position in space changes. Figure \ref{SinglePlayerObfuscation}a is
the true baseline, with a curve indicating the continuous true position, and the dots
representing reports of this position to the location-aware service. For each
strategy we discuss: \begin{inparaenum}[\itshape i\upshape)]
\item what kind of alteration of baseline data is performed;
\item what the motivation and possible use cases are;
\item how some form of computational support aids in the deception;
\item some of the systems which do this currently.
\end{inparaenum}

\subsubsection{Noise injection}

The most computationally simple form of obfuscation is the addition of noise to
the reports which are sent to the semi-trusted party. Here, the points which are
submitted deviate from the true values in a random manner. This allows the user
to conceal their exact location, while giving a broad indication of where they
are.

\todo{Why?} Avoid paparazzi/stalkers; conceal preferences - know you're in the
high street, but can't tell if you're buying texbooks in Waterstones or beer in
the White Horse

\subsubsection{Chaff}
\begin{itemize}
  \item Add in lots of extra points, with little relation to the real values
  \item Computationally easy
  \item Can include the real points
  \item Disturbs services which expect a single consistent trace
\end{itemize}

\subsubsection{Systematic Deviation}
\begin{itemize}
  \item Modify specific points
  \item e.g. hid that we're going to ballet lessons, and pretend we're at the
  boxing gym instead
  \item Computational support: Can use services to find most useful nearby
  places to pretend to be at; allows for automatic, thematic deviations
\end{itemize}

\subsubsection{Pretend to be me}
\begin{itemize}
  \item Generate ``user-plausible'' data, while the user deviates from their
  norm
  \item For instance, pretend to be at home while actually being on holiday
  \item CompSupp: have a model of normal behaviour which can be used to generate
  fake data
\end{itemize}

\subsubsection{Virtual Holidays}
\begin{itemize}
  \item want to pretend that we're deviation from normal when we're not
  \item i.e. pretend we've gone to a conference when we haven't
  \item CompSupp: create a theme and help manage the deception
\end{itemize}

\subsection{Multiplayer Obfuscation}
\fig{MultiPlayerObfuscation}{Multiplayer obfuscaction strategies: i) artificial
co-location; ii) supporting information; iii) account sharing}

\subsubsection{Artificial co-location}
\begin{itemize}
  \item We can pretend that we're in the same place as our friends
  \item CompSupp: need to coordinate lies with friends
\end{itemize}

\subsubsection{Supporting Evidence}
\begin{itemize}
  \item We ask our friends for supporting evidence of our lie; could be like
  co-location, could be broader
\end{itemize}

\subsubsection{Account Sharing}
\begin{itemize}
  \item A group of people share accounts, with some way to decide which is
  used, e.g. accounts covering geographic areas
  \item Like swapping supermarket discount cards
  \item CompSupp: which account, when? discovery and sharing etc.
\end{itemize}

\subsection{Verification and provenance mechanisms}

\fig{Verification}{Example verification scenario. The user provides a set of
real data, plus \emph{chaff} to a location aware service. A third party then
requests verification of some of the points, which the user provides}

Previously, we said verification is better. Why is this?

\begin{itemize}
  \item Lets assume that we've made our public data completely unreliable so noone can
use it.
\item If someone wants to engage with it, they have to talk to us
\item we can claim a subset of the data - just what they say they need for the
purpose at hand
\item We then have a choice of how to respond:
\item No signing: if we simply send them a message saying ``these
points belong to me'', then they can use the data, but would not be able to
convince third parties of its validity.
\item sign with our private key; now, anyone can check that we have claimed that set of datapoints
\item sign with our private key and their public key: they can't prove the data
is ours without revealing their identity.
\end{itemize}

\subsubsection{Notarization}
\fig{Notarization}{Notarization of personal data. a) Datapoints and times are
hashed, and the values sent to a notary service, which provides a URL to verify
that i) the given data was supplied and ii) when it was supplied. Hashes are
used so that the data is not publicly shared. b) If the hash of the previous
submission is included, then sequences of consecutive points can be verified.}

Third party notarization services can be employed. They take
in some datum, and provide a hash/URL which can be visited to verify that that piece of data was
provided at a certain time. For example, if someone wants to make a prediction
for the outcome of a football match, they could notarize that before the match,
and then subsequently prove that they had made the prediction beforehand. It is
generall not possible to prove that they only made a single prediction, however,
so this technique is most suitable when the range of possible things to notarize
is so large as to make notarizing the entire space infeasible.

With regard to personal data, we can notarize our true data stream as we produce
it. This means that we can prove that we had considered those points at the
time, and if we say we were in a particular place, there is a high chance we
were---however, it does not work in the complementary situtation as producing a
notarized point does not prove that we were not anywhere else.


Notarization does not necessitate revealing the data itself. For instance,
when submitting a location, a representation of the time and place could be
hashed, and this hash notarized (Figure \ref{fig:Notarization}a). Additionally,
points can be notarized in sequence, so that we can demonstrate contiguous sub-sequences of points as having been provided previously; by
hashing the current location with the previous location, we can link the points
together, to build up confidence in the notarized results (Figure
\ref{fig:Notarization}b).

\section{Conclusion}

\begin{itemize}
  \item Translation to things that aren't location data; generalisability; can't
  add chaff to our bank accounts (or can we?)
  \item Viability - how do services react when we fill them full of noise?
  Plausible versions of these techniques
\end{itemize}

%%%%%%%%%%% The bibliography starts:
%\begin{thebibliography}{99}
%\bibitem{r1}
%\end{thebibliography}

\bibliographystyle{abbrv}
\bibliography{palimpsests}


\end{document}
