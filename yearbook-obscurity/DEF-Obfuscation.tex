% PLEASE USE THIS FILE AS A TEMPLATE FOR THE PUBLICATION 
% Check file IOS-Book-Article.tex
%

\documentclass{IOS-Book-Article}     %[seceqn,secfloat,secthm]
%\usepackage{mathptmx}
%\usepackage[T1]{fontenc}
%\usepackage{times}%
%
%%%%%%%%%%% Put your definitions here
\usepackage{paralist}
\usepackage{graphicx}
\usepackage{color}
\usepackage{url}


%Make figures easier
\newcommand{\fig}[3][0.9]{
\begin{figure}[tp]
\begin{center}
\includegraphics[width=#1\textwidth]{figures/#2}
\caption{#3}
\label{fig:#2}
\end{center}
\end{figure}
}

\newcommand{\tbox}[3][red]{{
\color{#1}\noindent{
   \fbox{\scriptsize{ {\bf #2} \textsl{#3}}}
   \vspace{2pt}
}
}}


\definecolor{darkgreen}{rgb}{0,0.4,0}
\newcommand{\todo}[1]{\tbox{TODO:}{#1}}

%%%%%%%%%%% End of definitions
\begin{document}
\begin{frontmatter}          % The preamble begins here.
%
%\pretitle{}
\title{Social Palimpsests - clouding the lens of the personal panopticon}
\subtitle{\ldots or ``Suck my tailpipe'' - how I learned to stop worrying
and toxify my exhaust data}
%\subtitle{}

% Two or more authors:
\author[A]{\fnms{Dave} \snm{Murray-Rust}},
\author[B]{\fnms{Max} \snm{van Kleek}}
\author[B]{\fnms{Laura} \snm{Dragan}}
\runningauthor{Murray-Rust et. al.}
\address[A]{Centre for Intelligent Systems and Applications,
Department of Informatics, University of Edinburgh}
\address[B]{Southampton}

\begin{abstract}
The use of personal data has incredible potential to benefit both society and
individuals, through increased understanding of behaviour, communication and
support for emerging forms of socialisation and connectedness. However, there
are risks associated with disclosing personal information, and present systems
show a systematic asymmetry between the subjects of the data and those who
control and manage the way that data is propagated and used. In this chapter, we
explore a set of techniques for ameliorating the tension between the 
the benefits of sharing, and a distrust of those with whom we share our data.
\end{abstract}

\begin{keyword}
Obfuscation; Data politics; Personal Data Stores; Social Machines; Pants-on-fire;
\end{keyword}

\end{frontmatter}

%%%%%%%%%%% The article body starts:

\section*{Introduction}

%% maxxie wrote some stuff here

Privacy advocates have been accused of advocating methods to enable criminals,
stalkers, harassers \cite{}, miscreants, and even terrorists \cite{} to more
easily conduct their misdeeds against society. The ``I have nothing to hide;
therefore I should not care'' argument \cite{} used often in defense of ignoring
privacy in the early days of targeted advertising and the Web still re-surfaces
today. But as theoretical arguments of what \emph{may} become have been replaced
by tangible examples of what has actually has, this, among other arguments have
become thoroughly dismantled.  The sophisticated methods being used to track and
pinpoint details of individuals has not been met with a corresponding increase
in counter-tracking and counter-surveillance tools; as a result, people are
being subject to an unprecedented barrage of targeted attempts to behaviourally
manipulate them - often in ways too subtle for them to perceive (e.g.
\cite{facebookstudy}).           

Thus, this chapter takes the position that privacy protection not merely as a
concern for a minority with peculiar sensitivities towards being seen or
exposed, but instead, as a set of capabilities necessary for preserving
individual autonomy and well-being, in an economy where personal data has become
the most precious digital commodity. 

Bentham's thought experiment of the Panopticon, developed by Foucault describes
a prison in which inmates lives are constantly surveilled as a means of
discipline and exertion of control.
In the world of technologically driven data collection which we inhabit,
this has effects both
on people's behaviour as they internalise the fact that they are surveilled, and
on the way in which they are treated, and \emph{socially sorted} by the gears of
our algorithmic society\cite{simon2005Panopticism}.

We are interested in practises which can cloud the lenses of the observers, to
choose how we are seen, and to help regain some control of
the manner in which our lives are surveilled, both socially and otherwise:
\begin{quote}``We give up on privacy because we live in the Panopticon. A lot of
people think there is an inevitability about this [loss of privacy]. But the
availability of data does not sanction its use.'' Sir Nigel Shadbolt,
quoted in the Wall Street Journal\cite{rooney2012OpenData}\end{quote}

In this chapter, we discuss and analyse methods of privacy protection that
advance beyond the current state of anonymisation tools that merely obscure the tracks
of individuals towards those that employ methods borrowed from
information warfare \cite{}, in order to allow individuals to regain autonomy
from unsolicited tracking and behavioural control.  We first discuss X and Y, followed by a
survey of lying and falsification in context-aware systems, current
anonymisation ad privacy tools. This is followed by an overview of
strategies for obfuscation, some of which are currently implemented in either
mainstream tools or proof of concept studies,  and some of which are
speculative, future possibilities.

\section{Background}

\subsection{The rise of personal data and services reliant on it, and relation
so surveillance}

As we pass through the digitally augmented world that we collectively inhabit,
the set of actions with the potential to produce data grows year on year.
Portions of this outpouring are kept and stored as
\emph{capta}, from \emph{capere}: to keep \cite{dodge2005codes}. From
using an access card to unlock a door at the workplace, right down to tracking
individual footfalls, pervasive digital systems illuminate and annotate our
physical activity. Accreting around this body of physical observation is an
expanding sphere of mental observation and analysis. This can take the form of
active practises around recording mental states, such as journalling, but it can
also include computational inference, where frequency of posting on social
networks becomes an adjunct metric for connectedness, and search terms 
indicators of intent. As such, the modes of collection of this information can
range from explicit, user initiated submission of data, through consensual
background recording to invisible, asymmetric electronic surveillance.

The pervasiveness of computationally mediated interaction in modern social
life means that for much of this data ``refusal is not a practical option, as
data collection is an inherent condition of many essential societal
transactions''\cite{brunton2011vernacular}. This leads us to introduce the term 
\emph{fiat data}---when an
organisation uses its position to demand (by fiat) the disclosure of certain
information in return for use of its services. The term comes from a loose
analogy with \emph{fiat money}: currencies whose value is derived from the
mandate of a government\footnote{Fiat money is, by definition
\emph{inconvertible}, and \emph{intrisically useless}, and hence must derive
its value from external forces, typically government decree, status as legal
tender or by being the required currency of taxation. While the conditions of
inconvertibility and uselessness are not necessarily applicable to personal
data, the mechanism of central authority decree is a common defining factor}. 
If
one wants to interact socially on Facebook, one must pay the tax in personal data which they
demand. This can be contrasted with modern cryptocurrencies such as Bitcoin;
here, there is no central authority to backstop value and demand taxation.
Rather, the participants in the economy collectively decide what the units
of currency are worth; this would be a personal data economy in which
participants decided what and how they wanted to share.

Increasingly, in order to utilise services, we must provide our data to third
parties. This ranges from mobile phone numbers being required for Yahoo
accounts, to location data being shared with Foursquare or Grindr, to the NHS
adding personal health information to centralised databases. 


In some cases, this is a
necessary requirement for the service to be worthwhile, but in many cases it
represents an attempt by the organisation to create a monetizable product from
its users. There is a spectrum of approaches from outright demands 
to asking, cajoling or encouraging users to furnish their data.
Increasingly, \emph{gamification} is be used to manipulate
users into self-surveillance, by providing rewards---whether within the system
or through the promise of self-improvement---for activities which require the
sharing of data to function:
\begin{quote}``Literally, within an hour of waking up, I am playing at least two games that promise to help me become a more productive
worker and prolific writer. \ldots I want to suggest two things: 1) that gamification is a form of
surveillance; and 2) this surveillance is
pleasurable''\cite{whitson2013gaming}\end{quote}
Fitness apps, activity monitors, location based social networks require the
user to hand over their location data in return for the promise of increased
fitness, self awareness to the ability to connect with others. 
This user-driven data collection becomes a form of \emph{participatory surveillance}:
\begin{quote}
Online social networking can also be empowering for the user, as the monitoring and 
registration facilitates new ways of constructing identity, meeting friends and colleagues 
as well as socializing with strangers. This changes the role of the user from passive to 
active, since surveillance in this context offers opportunities to take action, seek 
information and communicate. 
\end{quote} \cite{albrechtslund2008online}

In summary, there are multitudinous situations where we are coerced, cajoled or
manipulated into sharing our personal data, and the cost of avoiding such
sharing is increasingly becoming untenable for large sections of the population
of the industrialised world.

\subsection{Sharing is a loss of control}

Sharing data, by definition, is the entrusting of other parties with
information; this necessarily involves relinquishing control over how it is
subsequently handled and disseminated. 
\todo{eMax: Do we want to mention DRM here as a futile attempt to control how data is used once it's shared?}
 \todo{There is so much more we can say right here about this.}  

There are many issues with sharing data, here we highlight four of them:
\begin{description}
  \item[Sharing is persistent, while situations evolve;] once data is shared,
  there is no technical means to revoke it. However, the context around its sharing and the
  organisations involved are subject to change. 
  A government may decide to share previously confidential data, as in the case of the recent \emph{care.data} fiasco in the UK; a company can be
bought and its assets acquired---the purchase of Moves by Facebook raised issues
around the terms and conditions of data handling companies; and even without
malice accidents can expose vast swathes of personal data, or court proceedings
may force private communications to become public - the Enron emails still
represent the largest publicly available corpus of private emails. Essentially,
once data is shared, the sharer has no control over what happens to it.

\item[Technology improves:] what is safe to share now may not be in the
future. Brad Templeton from the Electronic Freedom Foundation uses the analogy
of ``Time travelling robots from the future'': the information collected now
will be subjected to increasingly sophisticated analysis techniques as time
progresses, so the implications of sharing that information can be far beyond
expectations. For example, in the future, it may be possible to carry out facial
recognition on massive quantities of CCTV footage, and reconstruct the movements
of a large proportion of citizens. This corresponds to the surveillance
robots coming back in time and monitoring us now. In a similar vein: ``Would
you have liked to be gay 40 years ago in a monitored society? Or an enemy of J.
Edgar Hoover with modern tools in his hands?'' \cite{templetonWatched}. Sharing
data today cedes control to the entities of tomorrow, with their greatly
enhanced capabilities.
\item[De-identification doesn't work:] data is often shared subject to the
condition that it will only be shared in an \emph{anonymised} or
\emph{de-identified} form. As a highly public example, Netflix challenged the
public to create a better recommendation engine, based on a corpus of anonymised
viewing histories. Subsequently, it was shown that many records within the
database could be identified by comparison with publicly available
sources \cite{narayanan2008Deanon}, let alone access to other, non-public data.
Narayanan and Felten's recent report \cite{narayanan2014Deidentification} explains in a non-technical manner why
de-idenfication of data remains problematic. This is also highly dependant on
the data in question; location data is extremely difficult to anonymise, with
four datapoints being enough for re-identification in many cases \cite{}
\item[Databases can be joined:] as more databases of personal information
become available, whether publicly or privately, the possiblity to match, join,
correlate and shara data increases, and the effects of single points expand well
beyond the environment in which they were created or shared. In short, data are
held in \emph{leaky containers}: ``data move freely between different sectors of
society with the result that information from discrete contexts, e.g.,
 private life, work life and shopping, are being mixed rather than contained 
 separately.''\cite[p.37--44]{lyon2001surveillance}.
\end{description}

 \todo{What you are sort of getting at as ``rights'' to data and the lack of
regulation thereof; there is no notion of data ownership in any legisltative
form, nor is there a notion of course of maintenance and tansfer of such terms
of ownership - except in the most select of contextx, people have no legal
``right to even their own data''}

\subsection{The case for lying and the importance of anonymity}

White lies can aid in social network growth: \cite{iniguez2014Deception}

Most, if not all, social interactions 
involve both strategic omissions and various kinds of lies and
non-truths to manage the myriad conflicting social demands placed upon us. 

Butler Lies: \cite{hancock2009butler}

Translucence: ``What we say and do with another person depends on who, and how
many, are watching.'' - \cite{erickson2000Translucence}

Contrast with Transparent society \cite{brin1999transparent}; power imbalance
between parties.

\subsection{Multiple Identity}


A natural part of online life is the ability to tailor the persona we present to
different communities and contexts. An individual may want to disclose certain
things to their professional colleagues, while presenting differently to friends
and family or non-mainstream friend groups. 
\todo{Amy's Content creation on youtube \cite{guy2014ConstructedIdentity}}
\todo{Ben Dalton's thesis about Persona's throughout the agents \cite{dalton2013Pseudonymity}}

\subsection{Why doing it socially is difficult}

Sharing certain personal data is a barrier to anonymity and obfuscation; 
data which is rooted in physical fact provides multiple opportunities for
joining up otherwise separate databases. 
The lie maintenance required to avoid discovery may be trivial (``sorry, I’m hungry,
have to go!'') but may become significantly more complicated as lies extend over
time, and become woven into the social fabric. The ability to compare
multiple accounts of history---especially once the time travelling
robots are involved---means that dissonance within the social
fabric is more obvious than weaknesses in a single thread.


\subsection{Why verification and provenance are better than sharing}

Sharing is a crude mechanism. Once data has been shared, the originator can no
longer exert control over it, and must rely on the behaviour of the recipient,
which as noted may fail to meet user expectations. As dana boyd
notes: ``Any model of privacy that focuses on the control of information will
fail.''. This leads the teenagers that dana studies to
engage in \emph{social steganography}, manipulating messages so that ``Only 
those who are in the know have the necessary information to look for and interpret the information 
provided.''\cite{boyd2012Networked}. Strategies like this work when there is a
difference in understanding between the surveilled and the surveiller, and
collapses as soon as the comprehension barrier is removed.

Validation, however is a more subtle tool:
if a user’s personal dataset can be made sufficiently questionable as to be useless on its own,
then locus of control shifts to the user choosing to validate parts of the dataset,
which can be performed in a more nuanced,
contextualised manner. If a user is the final arbiter of trust, they can decide
to i) sign parts of their record, so that it is verified public fact; ii)
co-sign it with another entity, so either can  alsoverify it but not anyone 
else;
iii) verify it through an anonymous channel, so that the entity to whom they
provide verification cannot propagate the claim further. This verification can
be carried out entirely separately from the datastore itself, allowing for the
presentation of different datasets as valid  in different contexts, as well as
unorthodox methods such as using the Bitcoin blockchain to notarise datasets, so
that they can be verified in the future without revealing them as true at the
time.

\section{Review of current approaches and tools} 
\todo{Max to do some writing}

\todo{Refs to work in\ldots}
\begin{itemize}
  \item Marwick, public domain: \cite{marwick2012Public}
  \item Reigeluth, data traces: \cite{reigeluth2014Data}
  \item Beer, algorithms and power: \cite{beer2009Algorithm}
  \item Goldberg, public/virtual participation: \cite{goldberg2010Rethinking}
  \item Rauhofer, Future Proofing Privacy: \cite{rauhofer2012FutureProofing}
  \item Simon, panopticism \cite{simon2005Panopticism}
\end{itemize}


\begin{itemize}
  \item The revolution has started!
  \item tor, anonymous remailers, burner phones, gotta change up, yo!
  \item HTTPSEverywhere
  \item Surveillance and Society: \url{http://library.queensu.ca/ojs/index.php/surveillance-and-society}
  \item CVDazzle
  \item Heat-signature cloaking burqas, hoodies
  \item Bluetooth and MAC randomisation in iOS 8
  \item Silent Circle, Cryptocat
  \item DNT in IE10
  \item Adblock/Adblock Plus, Privacy Badger, Disconnect.Me
  \item HTTPsEverywhere
  \item VPNs 
\end{itemize}

Open source and trustworthiness

Theory of obfuscation:
Types of disinformation \cite{alexander2010Disinformation}:
\begin{description}
  \item[redaction] is hiding some or all of the information in a message
  \item[airbrushing] is changing some of the information. \emph{local crowd
  blending} means change it to a nearby message likely to be plausible.
  \emph{global crowd blending} means change it to a message in a dense part of
  the space.
  \item[curveball] add extra distracting information, push message into low
  density space
\end{description}

Some existing stuff and the things we can link it to later

\begin{itemize}
  \item TrackMeNot generates plausible google searches (Chaff)
  \item FaceCloak? Encrypts facebook data
  \item CacheCloak - sends a range of plausible future paths to location based
  services (Palimpsestification)
  \item Shopping card loyalty swaps (Account Sharing)
  \item DuckDuckGo - mixing up user searchers (Account Sharing, no cleverness)
  \item CVDazzle
\end{itemize}





\section{A selection of obfuscation strategies}
\label{sec:strategies}

Alexander's taxonomy \cite{alexander2010Disinformation} discusses several types
of disinformation which relate to modifying single messages. In contrast, due to
the pervasiveness of modern communications, we are concerned with modifying
message \emph{streams}, where a trace of multiple values must be considered.
The social aspect inherent to modern communication tools increases the 
[possibility of] interaction with others, which in turn increases the 
possibility of exposure \todo{what's that word i'm looking for?? detection/ 
yearbook-obscurity/DEF-Obfuscation.tex
uncover / disclosure/ reveal} of the lies. However, we can also use [the 
socially expected] social interaction to our advantage, colluding to strengthen 
the obfuscatory practices. yearbook-obscurity/DEF-Obfuscation.tex
% Additionally, there is the possibility of interaction with others, whether it is
% collusion to strengthen obfuscatory practices, or the addition---purposeful or
% otherwise---of information which exposes the 
obfuscation.yearbook-obscurity/DEF-Obfuscation.tex

In this section, we present a range of obfuscation strategies, some of which are
speculative, but many of which are drawn from existing examples both inside and
oustisde the digital sphere.

For each strategy we discuss: \begin{inparaenum}[\itshape i\upshape)]
\item what kind of alteration of baseline data is performed;
\item what the motivation and possible use cases are;
\item how some form of computational support aids in the deception;
\item how the strategy can be applied to other data
\item the systems (if any) which do this currently.
\end{inparaenum}


It is problematic to consider the obfuscatory tactics here without a sense of
the scenario in which they are to be deployed. Our scenario in this chapter is:
\begin{quote}
The user wishes to make use of services which expect location information; 
the location information provided is shared publicly and is almost
certainly stored indefinitely. At times, the user may want to draw on location
based information---such as restaurant recommendations or directions---and there
may be times when they wish to verify that they were at a particular location.
\end{quote}
The service is hence \emph{semi-trusted}: there are some benefits which the user
wishes to accrue, but there are aspects of the service which makes the user
unwilling to entrust their complete life history to it. We have chosen location
as a clearly understandable facet of personal data, and one which can be easily
used to re-identify individuals from anonymised
datasets\cite{montjoye2013Unique}.

\fig{Mediation}{Models of interaction with semitrusted services. a) Direct
transmission of information; b) computationally mediated transmission, where a
personal data store is enlisted to aid in obfuscatory processes.}

The standard model of interaction (Figure \ref{fig:Mediation}a) involves the
user submitting their data directly to the service; for our obfuscatory
techniques, we would like to enlist computational support (Figure
\ref{fig:Mediation}b). This typified, but not limited to mediation from a PDS
which acts on behalf of the user to modify the data which they provide.
In Figures \ref{fig:SinglePlayerObfuscation} and
\ref{fig:MultiPlayerObfuscation}, we plot a fictitious one-dimensional
``location" measurement against time in order to give a sense of how
obfuscations unfold across time in multiple locations\footnote{While a two
dimensional, map-based representation would be more immediate, it is difficult
to clearly show temporal aspects.}. We show the individual's true `location' as
a continuous line, along with the measurements made by their device; we then
overlay the points which would be submitted on their behalf after obfuscation.

\subsection{Strategies for the lone obfuscator}

\fig[1.05]{SinglePlayerObfuscation}{Obfuscation strategies for the lone agent}

Figure \ref{fig:SinglePlayerObfuscation} lists a collection of possible
obfuscation strategies. In all cases, a fictitious one-dimensional ``location"
measurement is plotted against time, to give a sense of how an
individual's position in space changes. Figure
\ref{fig:SinglePlayerObfuscation}a is the true baseline, with a 
curve indicating the continuous true position, and the dots
representing reports of this position to the location-aware service. For each
strategy we discuss: \begin{inparaenum}[\itshape i\upshape)]
\item what kind of alteration of baseline data is performed;
\item what the motivation and possible use cases are;\todo{I would split this 
in 2 - motivation, then use cases}
\item how some form of computational support aids in the deception;
\item how the strategy can be applied to other data \todo{do we no do this in 
the next section for all strategies?}
\item some of the systems which do this currently.
\end{inparaenum}

\subsubsection{Chaff}

World War II fighter planes would emit clouds of radar reflective
sheets---\emph{chaff}---which created multiple traces the screens of radar
operators, and hence disguised the true position of the aircraft. In a similar
manner, we can add in multiple location datapoints alongside the real ones. This
is the one of the few methods where the complete, accurate datastream is stored.
Hence the user can still access any benefits which rely on accurate information. However,
adding a multitude of randomised points to a service which expects a single
contiguous trace is both easily detectable, and may break functionality---a
run tracking application would be likely to give unreliable distance estimations
in the presence of chaff.

\subsubsection{Noise injection}

The most computationally simple form of obfuscation is the addition of noise to
the reports which are sent to the semi-trusted party. Here, the points which are
submitted deviate from the true values in a random manner. This allows the user
to conceal their exact location, while giving a broad indication of where they
are. Depending on the level of noise, this can allow the use of location
based services without revealing much about actual behaviour. For example, it
might reveal your location on the high street so you can arrange to meet
friends, without revealing which shops you were visiting. This is compatible
with services which expect coherent location data, and may be indistinguishable
from the inaccuracies of the location sensors. One downside is that the ``true''
location traces are not present in the record of the service.
% \todo{examples?}
For example, TripAdvisor can still provide a good enough list of recommended 
attractions around the given ``noisy'' location, however a navigation 
application will not be able to provide reliable directions.

\subsubsection{Coarsened Granularity (or Quantisation)}

Rather than adding noise to the data being sent, it can instead be quantised to
a coarser granularity, akin to blurring, or zooming out on a map. Again, this is
a technique which may help to derive useful information from the service,
without revealing more than is necessary: using a service to find friends in the
same city should only require city level information to be shared. An example of
this can be found in Android's permission system, which has separate controls
for \verb|ACCESS_COARSE_LOCATION| versus \verb|ACCESS_FINE_LOCATION|; similarly,
posted letters may be signed with a city rather than a street address.

\subsubsection{Systematic Deviation}

In some cases, it may be possible to introduce systematic deviations into the
digital record. In order to this, the user needs to be able to define which
points to alter, and what to replace them with. One possibility would be
thematic replacement---``hide the times I went to the pub by saying I
was at a cafe''. Another would be to disguise the user's home and work
locations---places where they are less likely to require location based
searches, but which make it very easy to re-identify them from anonymised data.
It is likely that this will require some form of computational support to i) identify targets for replacement as they occur and ii) find suitable replacements. Using this technique, some, but not all of the true data is stored; however derived information---such as beverage preferences in the
example above---can be wildly and purposefully distorted. The nature and fact of
the distortions may be hard to uncover, as no simultaneous traces or
strange movement patterns are produced. Depending on the domain, subtle
alterations may have large effects.
\todo{examples?}

\subsubsection{Pretend to be me}

With increasing computational support, it becomes possible to create a model of
the user which outputs plausible ``normal'' data. This can then be used to
replace periods of abnormal behaviour, or even replace normal behaviour with
statistically similar but untrue data. An early example is when neighbours (or
automatic switches) are employed to turn lights on and off in a home which has
been vacated for the holiday, disguising the true anomalous data of a dark,
empty house with the appearance of normal occupation. Similarly, one might avoid
making Facebook posts which indicate an absence, to avoid burglary. This kind of
deception can be difficult to achieve; however computational systems are
emerging which can aid users, for example Beyer's digital alibi system
\cite{beyer2014Alibi}.

\subsubsection{Coherent Deviation}
As the converse of simulating normality, the user may wish to pretend to be
somewhere where they are not \todo{more motivation for Alibot!}. This is similar
to creating systematic deviations, but on a grander scale; the user would like
to create a narrative for the deviation, and then have suitable data points
constructed. For example, the use might pretend to be on holiday, or at a
conference, and would like location traces which match that narrative to be
created, such as going to the convention centre in the day, and returning to a
hotel at night. This requires a computational model of user behaviour which can
be applied to new locations---a non trivial task. However there is the potential
to create obfuscated data which is difficult to distinguish from standard
behaviour. \todo{examples}

\subsubsection{Palimpsestification}
Taking the idea of coherent deviations a stage further, and combining with the
idea of \emph{chaff}, the user could create multiple overlapping traces; each
trace would be locally coherent and plausible, but someone inspecting the data would have no way to know which is the real one.
This is similar to the strategy of CacheCloak \todo{reference}, which
continuously generates sheaves of probable future behaviour and searches
location based services relevant to each  path. The computational support
required is similar to the coherent deviation example---to be able to run a
model of the user's behaviour in novel locations---although more coordination
might be required between the stories. The tradeoff is that while the true
location data can be entered along with the generated points, the deception is
obvious, and location based services may become upset at the multiple paths.

\subsection{Collaborative Obfuscation}
\fig{MultiPlayerObfuscation}{Multiplayer obfuscaction strategies: i) artificial
co-location; ii) supporting information; iii) account sharing}

Including others in the obfuscation challenge opens up a range of new 
strategies, where collusion can aid in the creation of otherwise unachievable
datastreams, or increase the veracity of artificially created data. Generally
the possibilities in computational systems are analagous to pre-computational
possibilities; computational support tends to be in the form of coordination to
find collaborators or and check coherence of data points. The ideas outlined
here are more speculative, as few computational systems of this type exist.
There are aspects which make these strategies harder to pull off - coherence is
then required across multiple different accounts; however the counterpoint is
that if successful, the obfuscation is better supported and harder to spot.

\subsubsection{Artificial co-location}
One way to obtain a realistic but untrue location trace is to re-present the
trace of a collaborator. This can look like relatively natural
behaviour; two people meeting up to carry out joint activities or socialisation.
Computational support here can involve finding accomplices to ``co-locate''
with---people who are willing to share their location, and are behaving in ways
which match the desired story---as well as the technical business of
transferring location devices between accounts.

\subsubsection{Supporting Evidence}

\todo{Maybe merge into preceding co-location section}
Co-located people often share the fact of their co-location, explicitly or
implicitly, whether in group photos---``Here's me and X on top of the Scott
monument''; broadcast messages---``Just been hanging out with X at the coffee
shop''; or shared plans---``Going to the movies with X tonight - anyone want to
join in?''. Enlisting collaborators to make these kinds of posts can help to add depth to a
constructed trace, weaving it more tightly into the social fabric.

\subsubsection{Account Sharing}

In a similar manner to the swapping of loyalty cards discussed in
\cite{brunton2011vernacular}, users of services can share accounts. This
results in an account or set of accounts with more or less plausible activity,
yet allows the users to remain unidentified. 
Much as the loyalty card swapper confounded efforts to inspect individual
buying habits, or the ``Anonymous'' movement aggregates the activities of a
multitude of participants behind a stylised mask, services such as DuckDuckGo
aggregate many people's search results, ensuring that the search providers
cannot build up any identifiable user histories. 

Computational intelligence can be enlisted to support many different ways of assigning people to accounts,
 such as:
\begin{description}
  \item[\emph{Many to one}] schemes have a single account controlled by
  multiple people. This can result in a completely incoherent manner;
  DuckDuckGo's aggregated search makes no attempt to imitate
  individual behaviour. 
  Alternatively, sharing can be tied into a coherent shared identity,
   where multiple people contribute to a single shared persona\cite{dalton2013Pseudonymity}. 
   Here there is a challenge to maintain consistency: when multiple people
   control a call centre's chat avatar, they must ensure that the relevant
   information and state is shared [\emph{ibid.}]. When multiple users control a
   single game character, the gameworld enforces consistency, and the community
   must produce coherent action streams in order to progress \todo{cite
   Twitch!}.
   \item[\emph{Randomised}] schemes allocate accounts to people without a
   guiding principle; when loyalty cards are mailed between anonymous
   participants, there is an explicit desire to produce implausible data in
   order to confound analysis. Online accounts can be similarly shared, leading
   to traces which are unlikely to have been produced by a single individual.
   In our locative service example, this allows users to access benefits which
   do not rely on individual history, while preserving some level of privacy.
   Computational support involves finding accounts to share, and ensuring that
   each account is only accessed by a single person at any given time.
   \item[\emph{Structured}] schemes allow for accounts to be used as appropriate
   according to some criterion. If a location service offers history based
   benefits (e.g. loyalty rewards or reputation) then it could be beneficial to
   borrow a local user's account when going on holiday---Couchsurfing but with
   login credentials instead of flats. Infrastructure would be required to
   discover appropriate accounts, and mediate access. \todo{More explanation? Link to powerlevelling
   in WoW?}
\end{description}

\section{Operationalisation - managing deception and its side effects}

\subsection{Going beyond location}

In Section \ref{sec:strategies}, we discussed obfuscatory possibilities with
respect to a location based service; however, this is a single application area,
and the need for regaining informational autonomy is felt across spectrum of
datatypes and services, hence we must discuss how these techniques generalise. 

Location data is generally collected by a device which the user owns. In many
cases, this is a smartphone, which uses a combination of GPS, cell tower
triangulation and WiFi access point locations to determine a user's position in
space. The user then has some level of choice about who to share the data with
and how. This is not always the case, however: cell tower records can identify
user's locations---and individuals can be picked out from very sparse histories
\todo{cite Nature article on cell location identification}.


\subsection{PDSs to support obfusction}
\begin{itemize}
  \item create continuity
  \item act on your behalf
\end{itemize}

\subsection{Personal Data Stores - allies on the intimacy battleground}

Personal data stores (PDS) represent a partial solution to issue of
presentation: having trusted, user controlled repositories for data enables a
more user-centric approach to management of capta---those data which we choose
to take and preserve. Bridges can then be built between personal data stores and
the rest of the world in order to support the connected, networked interactions
which users now expect. If these bridges simply share the data, even in a
controlled manner, nothing has been gained; hence the bridges become conduits
for manipulating truth and constructing falsehoods. As personal data stores
accumulate more real-time contextual data about the individual, as well as about
the individual’s social connections, PDSes can provide support for the often
stressful and mentally burdensome task of lie maintenance, for example: i)
identifying when a person's real activities or whereabouts contradict a lie, and
might be discovered; ii) identifying indirect social channels that could expose
a lie (e.g. through friends of friends); iii) suggesting appropriate lies to use
which are least likely to be detected; iv) suggesting individuals to lie to to
support lie maintenance (e.g. friends of the person being lied to)


\subsection{Verification and provenance mechanisms}

\fig{Verification}{Example verification scenario. The user (Ally) provides a set
of real data, plus \emph{chaff} to a location aware service. A third party
(Brett) then requests verification of some of the points, which Ally
provides. Brett then wishes to share the data with Charlie, which requires Brett
to verify to Charlie that the data are correct.}

In the introduction we suggested that verification is a more nuanced mechanism
than control over sharing, since sharing is impossible to control
technologically. One of the effects of the obfuscation strategies discussed
previously is that it becomes impossible to know which parts of the user's
data-stream are grounded in reality, and represent ``true'' values. This means
that if someone wishes to engage with the data and have an expectation of
accuracy, they need to ascertain which parts of the record are correct. This
shifts the locus of control from the process of sharing to the process of
verification---the user can make claims about subsets of the datapoints
currently attributed to them.

Let us consider a scenario where Ally has some personal data, which Brett would
like to make use of. Brett also wants to sell Ally's data to Charlie.

There are a range of statements which Ally can make, 
including:
``this subset of datapoints is mine'',
``these points are within 50m of my true location'',
``these points are representative of my general behaviour'' and so on. 
The choice of which point to claim can be negotiated
in the context of the question being asked, and Ally can determine what is
and is not acceptable.

If the external agency wishes to disseminate the users data, it becomes an issue
of propagating the trust which the user has given them---essentially, Brett must
say to Charlie: ``Ally has verified these points to me, and now I am
verifying them to you''. The manner in which the initial verification was carried out now becomes
critical:
\begin{itemize}
  \item if an email or similar communication is used, Ally simply declares
  ``these points are mine'', then the secondary verification is only as strong as
  trust in the communication chain---Brett must convince Charlie that the email
  or message is genuine and emails are easy to fake;
  \item Ally could use a technique which would give Brett no future tangible
  proof of the verification---for example, a single use URL which
  lists IDs for the correct points. Brett would have no evidence with which to
  convince Charlie that Ally had verified the points, other than reputation
  alone.
  \item Ally can cryptographicaly sign the claim using public key cryptography.
  The claim is then essentially public knowledge, and anyone can check Ally's verification.
  \item Ally can sign the claim after Brett has; this means that Brett cannot
  hide the fact that they were the recipient of the claim, so it is impossible
  to propagate the claim anonymously.
\end{itemize}

All of these techniques relate to making the public record so unreliable that
anyone who wants to use any of the data will need to separately establish a
chain of provenance for certain parts of it. A related goal would be to make it
illegal, or at least unacceptable, to use personal data without having a valid
provenance chain for it. Essentially, in order to use anyone's data, Charlie
would have to explain how they came to have it, and be able to prove that Ally
had shared the data originally.


\subsubsection{Notarization}
\fig{Notarization}{Notarization of personal data. a) Datapoints and times are
hashed, and the values sent to a notary service, which provides a URL to verify
that i) the given data was supplied and ii) when it was supplied. Hashes are
used so that the data is not publicly shared. b) If the hash of the previous
submission is included, then sequences of consecutive points can be verified.}

The verification examples above rely on Brett trusting Ally about which
datapoints are correct. There may be times---e.g. when creating alibis---when
Ally needs to have a stronger form of proof.

In this case, third party digital notarization services can be
employed\footnote{e.g.
\url{http://virtual-notary.org/}, a free service hosted at Cornell University}.
These services take in some document or datum, and provide a certificate which
can be used to verify that that piece of data was provided at a certain time. 
For example, if someone wants to make a prediction
for the outcome of a football match, they could notarize that before the match,
and then subsequently prove that they had made the prediction beforehand. It is
generally not possible to prove that they only made a single prediction,
however, so this technique is most suitable when the range of possible things to notarize
is so large as to make notarizing the entire space infeasible.

With regard to personal data, we can notarize our true data stream as we produce
it. This means that we can prove that we had considered those points at the
time, and if we say we were in a particular place, there is a high chance we
were---however, it does not work in the complementary situtation as producing a
notarized point does not prove that we were not anywhere else.


Notarization does not necessitate revealing the data itself. For instance,
when submitting a location, a representation of the time and place could be
hashed, and this hash notarized (Figure \ref{fig:Notarization}a). Additionally,
points can be notarized in sequence, so that we can demonstrate contiguous sub-sequences 
of points as having been provided previously; by
hashing the current location with the previous location, we can link the points
together, to build up confidence in the notarized results (Figure
\ref{fig:Notarization}b).

\section{Conclusion}

\begin{itemize}
  \item Translation to things that aren't location data; generalisability; can't
  add chaff to our bank accounts (or can we?)
  \item Viability - how do services react when we fill them full of noise?
  Plausible versions of these techniques
  \item Ethics - is this OK?
  \item Obfuscation evolves in lockstep with systems to see through it; future
  people will be better at spotting constructed points.
  \item In the short term services will start to become more suspicious about
  the data that goes into them; start rejecting points which represent causality
  violations.
\end{itemize}

%%%%%%%%%%% The bibliography starts:
%\begin{thebibliography}{99}
%\bibitem{r1}
%\end{thebibliography}

\bibliographystyle{abbrv}
\bibliography{palimpsests}


\end{document}
